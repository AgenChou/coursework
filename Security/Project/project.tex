\documentclass[11pt,journal]{article}
%\usepackage{hyperref}
%\usepackage[breaklinks]{hyperref}
\usepackage{breakurl}
\usepackage{url}
\usepackage{listings}
\usepackage{courier}
\usepackage{amsmath}
\usepackage{graphicx}
\graphicspath{ {/home/agata/Documents/coursework/SensorNetworks/Lab4/} }
%\ifCLASSOPTIONcompsoc
% IEEE Computer Society needs nocompress option
% requires cite.sty v4.0 or later (November 2003)
\usepackage[nocompress]{cite}

%\else
% normal IEEE
\usepackage{cite}
%\fi

\hyphenation{op-tical net-works semi-conduc-tor}
\addtolength{\oddsidemargin}{-.875in}
\addtolength{\evensidemargin}{-.875in} 
\addtolength{\textwidth}{1.75in}

\addtolength{\topmargin}{-.875in}
\addtolength{\textheight}{1.75in}

\begin{document}
	\title{CS915 Advanced Computer Security, Final Project}
	
	\author{UID: 1690550}% <-this % stops a space
		%\protect\\
		%\thanks{}}
	
	% The paper headers



	% IEEEtran.cls defaults to using nonbold math in the Abstract.
	% This preserves the distinction between vectors and scalars. However,
	% if the journal you are submitting to favors bold math in the abstract,
	% then you can use LaTeX's standard command \boldmath at the very start
	% of the abstract to achieve this. Many IEEE journals frown on math
	% in the abstract anyway. In particular, the Computer Society does
	% not want either math or citations to appear in the abstract.
	
	% Note that keywords are not normally used for peerreview papers.
	
	% make the title area
	\maketitle
	
	
	% To allow for easy dual compilation without having to reenter the
	% abstract/keywords data, the \IEEEcompsoctitleabstractindextext text will
	% not be used in maketitle, but will appear (i.e., to be "transported")
	% here as \IEEEdisplaynotcompsoctitleabstractindextext when compsoc mode
	% is not selected <OR> if conference mode is selected - because compsoc
	% conference papers position the abstract like regular (non-compsoc)
	% papers do!
	%\IEEEdisplaynotcompsoctitleabstractindextext
	% \IEEEdisplaynotcompsoctitleabstractindextext has no effect when using
	% compsoc under a non-conference mode.
	
	
	% For peer review papers, you can put extra information on the cover
	% page as needed:
	% \ifCLASSOPTIONpeerreview
	% \begin{center} \bfseries EDICS Category: 3-BBND \end{center}
	% \fi
	%
	% For peerreview papers, this IEEEtran command inserts a page break and
	% creates the second title. It will be ignored for other modes.
	%\IEEEpeerreviewmaketitle
	\section{Part 1}
	\subsection{Cross-site scripting}
	Cross-site scripting (XSS) occurs when a user injects a client-side code into a web page, through a field taking user input. This code is then executed in another's client browser (as is the case with DOM-based and reflected attacks) or on the server side (stored attacks). It occurs when the website fails to validate user's input. It is considered to be of high, although not critical severity\cite{xss review}. It is the most common type of website attacks, which has been around since 1990s\cite{xss2}.
	
	The aim of XSS attacks is most often to access a client's usernames and passwords, and other data such as banking information, which is often stored in the browser.
	
	A real-life example of an XSS attack is a malicious script known as "Samy Worm", which in 2005 propagated across MySpace, making over 1 million users execute the payload within just 20 h. The  development of the worm and the details of the implementation were described by the author himself after MySpace has fixed its issues, and is still available online: \url{http://samy.pl/popular/tech.html}.
	
	In the case of our website, we can perform an XSS attack on the Guestbook page. Posting a comment with the content as follows:
	
	\texttt{<script>alert("Hello!")</script>}
	
	Will cause this comment to be saved to the database as is and in
	
	This vulnerability is found whenever we take user input and post it to the database, which is then displayed for other users.
	
	To prevent this sort of attacks, we need to strip user input of tags. For example, PHP function \texttt{strip\_tags()} is capable of that. So instead of \texttt{\$row['name']} and \texttt{\$row['comment]} we should have \texttt{strip\_tags(\$row['name'])} and \texttt{[strip\_tags(\$row['comment']}.
	
	\subsection{Directory traversal attacks}
	Directory traversal attacks aim to access the parent directory and traverse the directory tree on the server, by providing user input which includes multiple \texttt{../} that traverse back to the root directory. Note that repeating \texttt{../} more times than necessary after reaching the root directory does not have any effect.
	
	It can be used to read or execute files stored on the server. The most common PoC of directory traversal is accessing \texttt{/etc/passwd}, where the passwords are stored on a Unix system. We can do it on the website by going to the following address:
	
	\texttt{127.0.0.1:4567/cats.php?page=/etc/passwd}
	
	We can also some files that we don't see on the website, for example:
	
	\texttt{127.0.01:4567/cgi-bin/basic.cgi?file=basic.cgi}
	
	A way to prevent it is to strip '../' or '\%2e\%2e\%2f' or any combination of them and do not store any files that are not necessary for the website in the website's directory.
	
	
	\subsection{SQL code injection}
	SQL code injection relies on inserting SQL code devised by the attacker into the server's database. The code may involve for example dropping the tables. It is another example of an attack which can be prevented by sanitizing user input correctly.
	
	Some of the most severe cases of SQL injection attacks involve an American hacker Albert Gonzalez, who in 2005-2007 used this technique on corporate systems to steal and later re-sell over 1.5 million credit card details\cite{Gonzalez}.
	
	This can be done easily on the guestbook page by adding the comment: 
	
	\texttt{first comment'),(now(), 'Eve', 'second comment')}
	
	It results in adding two comments instead of one.
	
	There are many methods of preventing SQL injections. It can be done by adjusting privileges, or by simply sanitizing input, e.g. to escape all characters such as quotes, '\%' or '*'.
	
	
	\subsection{Python code injection}
	This time we are interested in injecting Python script into the website. In our website, the function plotter is written in Python and at line 30 of the file 'domain.cgi' we see the following statement:
	
	\texttt{v = eval( function )}
	
	Where \texttt{function} is unsanitized user input. Now if we input a function such as:
	
	\texttt{print("\textbackslash r \textbackslash n"+ open('/root/website/cgi-bin/basic.cgi', 'r').read() and sys.exit(0)}
	
	Python's \texttt{eval()} function is a known vulnerability, and we should avoid it whenever possible. Unfortunately, this is not easy in this case, so we have to make sure it can evaluate safely by using additional parameters. We need to define a dctionary of safe function, for example \texttt{safe\_dict = \{"x" : x, "y" : y, "tan" : tan\}}, and then change line 30 of 'domain.cgi' to:
	
	\texttt{v = eval(function, "\_\_builtins\_\_":None, safe\_dict)}
	
	\subsection{Stack-based buffer overflow}
	C is especially susceptible to stack-based buffer overflows. These types of attack rely on no bound-checking, thus allowing an attacker to exceed buffer size. It usually involves overwriting the memory that was not intended to be accessed, but can also mean reading it, thus revealing sensitive information.
	
	The best-known example of buffer overflow attacks is the Heartbleed bug present in the OpenSSL library. It was exploited in 2014, when hundreds of social insurance numbers were stolen from the Canada Revenue Agency\cite{heartbleed news}.
	
	Note that in the RPN calculator, we are allocated a buffer of size 2048 for terms in the calculation. Therefore we can exceed the buffer size by writing '2 ' 1026 times followed by '- ' 1025 times, which causes stack overflow.
	
	To prevent this, we can change the condition of the while loop in the main function of the RPN to be \texttt{while (pptr != NULL \&\& !finished \&\& stack\_top < MAX\_TERMS) }
	
	\subsection{Heap-based buffer overflow}
	Heap-based buffer overflow differs from a stack-based one in that it relies on dynamic memory allocation, and usually aims to corrupt the application's data, as reading or precise overwriting of data is much more difficult than in a stack overflow. 
	
	One of the more severe cases of such an attack was exploiting a bug in Internet Explorer's XML parser in 2014. It allowed for execution of an attacker's code in the browser. It is not the first issue of this kind with Internet Explorer. 
	
	In the RPN calculator's header, \texttt{context.h} we are allocated 32 characters for a variable name using \texttt{malloc}, so if the user input contains a longer variable name, we get a buffer overflow. 
	
	To fix it, we need to check the length of the variable. Additionally I would prefer not to use dynamic memory allocation, and have all buffers of fixed size.
	
	\subsection{Format string vulnerability}
	A format string attack occurs when a user-submitted string contains specifierssuch as \texttt{\%s} or \texttt{\%d}. It may allow the attacker to read the content of the stack or perform a denial of service attack.
	
	A notable case of an exploit of this vulnerability occurred in 2000, when hackers gained root access to a host running an FTP daemon\cite{format string}.
	
	In our website this vulnerability is present in the RPN calculator. More precisely, as soon as the \texttt{interpret()} function sees '//', it does not read past it into the content of the comment. It simply prints the comment with \texttt{printf()} which accepts specifiers.
	
	For example, changing the first line of the default content of the calculator (the one we see when we first visit the page) to:
	
	\texttt{// RPN Calculator Syntax (This is a comment)\%s \%f \%s \%f}
	
	Results in a successful compilation and the following output:
	
	\texttt{[0.000000] // RPN Calculator Syntax (This is a comment)[\%f] \%s 0.000000  // RPN Calculator Syntax (This is a comment)\%s \%f \%s \%f 0.000000}
	
	(I suppose on the bright side we did get our original comment out at some point after all...)
	
	To prevent this attack, we need to inspect comment strings closer and strip it of format string specifiers. There are various ways to do that, for example replacing each '\%' with '\% '- i.e. '\%'followed by a white space.
	
	
	
	\section{Part 2}
	%\IEEEPARstart{}{} 
	
	First to prove that we can create and remove files from another machine, we turn our Python injection from earlier into a curl script which can be executed from a terminal. To this end we can notice that when the output of a function plotter is displayed, the address changes to \texttt{localhost/cgi-bin/domain.cgi}. We can execute this function from address bar by adding \texttt{?function=[python injection goes here]}.
	
	The vulnerabilities that allow for that are directory traversal and python injection, but especially the former.
	
	\subsection{Bell-LaPadula policy for file reading}
	We begin with file reading, using the security conditions of BLP, summarised as "no read up". We define set os subjects $S = {\text{root}, \text{server}, \text{MySQL server}, \text{user}}$ and the set of objects to be all files.
	
	We further define security levels $I_0$, $I_1$, $I_2$, $I_3$. Root the has highest level, $I_3$, thus being able to read all files on the disc. The web server (www-data daemon) has clearance $I_2$ and the MySQL server $I_1$. Finally the user has clearance $I_0$. 
	
	This means that the root can read all files. The security level of the website server ($I_2$) is defined to restrict read access to \texttt{/var}. MySQL server reads only \texttt{/var/lib}. Finally the user does not read any files. Since \texttt{/var/lib} is a subdirectory in \texttt{/var}, the website server can read the database of comments.
	
	\subsection{Biba's Strict Integrity Policy}
	
	Biba's policy operates on the principle "no read down, no write up". Here we are interested only in writing files. With security levels and subjects defined as before, we want to prevent lower level security subjects to write up.
	
	Then root can write to the server and execute the www daemon. The server can query and write to MySQL server (i.e. execute) because the security level of the MySQL server is lower than that of the website server, and similarly execute things in the user's browser. However user cannot directly write to neither the server nor the MySQL server now.
	
	We need to define an additional subject, a transport layer at a security clearance lower than both the server and the user. This way the server and the user can communicate - the server writes to the transport layer, and the user reads from it.
	
	This also ensures tranquillity, because nothing can write up, so no files get contaminated, and there is no need to downgrade a subject. Tranquillity is necessary for the normal operation of a website, because for example if the www server had the user writing to it, it would get downgraded and would not be able to read its own PHP files anymore.
	
	\subsection{Clark-Wilson}
	Note that the files intrinsic to the website's operation can be found in two places: in \texttt{/var/www} we have files used by the www daemon, and in \texttt{/root/website} we have a copy of the same files. Also note that shell scripts are in the parent directory of the latter. Therefore, let the user be able to write/delete files in the \texttt{/root/website} directory. This excludes sensitive data such as passwords.
	
	Constrained data items will be in the \texttt{/var/ww} directory, while the unconstrained ones in \texttt{/root/website}. To allow user to write to the latter, we need create a link such that whenever user attempts to write to disk, it is redirected to the latter directory. We only allow file types: .php, .html, .txt, and image files (i.e. we are specifically avoiding executables. Script will be stripped off files later).
	
	The transaction procedure (TP1) for writing to files is as follows:
	\begin{enumerate}
		\item User writes a file, which is stored in /texttt{/root/website}
		\item If the files are of a type that isn't allowed, the audit fails
		\item The website with the new file(s) is simulated in a forked shell
		\item If the website attempts to write to a different directory than its own, the audit fails
		\item If during the simulation, a file of a type that isn't allowed gets created, the audit fails
		\item Else the simulation ran safely, so we can copy the files across to \texttt{/var/www}
	\end{enumerate}

	Steps 2-5 describe the Integrity Verification Procedure, all of which gets logged. The separation of duty is ensured by the triple: website user - the above transaction procedure - files in \texttt{/var/www}. User cannot write to the CDIs without the transaction procedure, because all attempts will be redirected to \texttt{/root/website}.
	
	To deal with low integrity data, the TP1 will reject anything with a file type that is potentially harmful.
	
	In the case of reading files, we can simply allow user to read any file in \texttt{/var/www}, assuming that only files intrinsic to the website's operation are there. However we can see that the file basic.cgi is not essential, so it may be better to simply keep a list of files that are necessary for the website to function.
	
	The transaction procedure (TP2) for reading files is as follows:
	\begin{enumerate}
		\item user requests read access to a file
		\item if the file is in \texttt{/var/www} (or on the list), user gets read access
		\item otherwise we reject, and log
	\end{enumerate}
	
	
	
	
	
	
\begin{thebibliography}{1}
	
	\bibitem{xss review}
	I. Hydara et al., "Current state of research on cross-site scripting (XSS) - A systematic literature review" in \emph{Information and Software Technology}, vol. 58, 2015, pp 170-186
	
	\bibitem{xss2}
	B. Gupta et al., "Cross-Site Scripting (XSS) Abuse and Defense: Exploitation on Several Testing Bed Environments and Its Defense" in \emph{Journal on Information Privacy and Security}, vol. 11, 2015
	
	\bibitem{Gonzalez}
	S. Gaudin, "Government informant is called kingpin of largest U.S. data breaches" in \emph{ComputerWorld}, available:\\ \url{http://www.computerworld.com/article/2527161/government-it/government-informant-is-called-kingpin-of-largest-u-s--data-breaches.html}
	
	\bibitem{heartbleed news}
	P. Evans, "Heartbleed bug: RCMP asked Revenue Canada to delay news of SIN thefts", CBC News, April 2014, available: \url{http://www.cbc.ca/news/business/heartbleed-bug-rcmp-asked-revenue-canada-to-delay-news-of-sin-thefts-1.2609192}
	
	\bibitem{IE heap}
	"Microsoft Internet Explorer Heap Buffer Overflow Remote Code Execution Vulnerability", Zero-Day Initiative, April 2014, available: \url{http://www.zerodayinitiative.com/advisories/ZDI-14-034/}
	
	
	\cite{Kevin London}
	K. London, "Dangerous Python Functions, Part 1" in \emph{Kevin London's blog}, 26 June 2015, available: \emph{https://www.kevinlondon.com/2015/07/26/dangerous-python-functions.html}
	
	\bibitem{format string}
	AJ Kumar, "Format String Bug Exploration", InfoSec Institute, May 2015, available: \url{http://resources.infosecinstitute.com/format-string-bug-exploration/#gref}
\end{thebibliography}
	
	% that's all folks
\end{document}

