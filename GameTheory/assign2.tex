\documentclass[11pt,journal]{article}
%\usepackage{hyperref}
%\usepackage[breaklinks]{hyperref}
\usepackage{breakurl}
\usepackage{url}
\usepackage{listings}
\usepackage{courier}
\usepackage{amsmath}
%\ifCLASSOPTIONcompsoc
% IEEE Computer Society needs nocompress option
% requires cite.sty v4.0 or later (November 2003)
\usepackage[nocompress]{cite}

%\else
% normal IEEE
\usepackage{cite}
%\fi

\hyphenation{op-tical net-works semi-conduc-tor}
\addtolength{\oddsidemargin}{-.875in}
\addtolength{\evensidemargin}{-.875in} 
\addtolength{\textwidth}{1.75in}

\addtolength{\topmargin}{-.875in}
\addtolength{\textheight}{1.75in}

\begin{document}
	\title{Algorithmic Game Theory, Assignment 1}
	
	\author{Agata~Borkowska,~UID: 1690550}% <-this % stops a space
		%\protect\\
		%\thanks{}}
	
	% The paper headers



	% IEEEtran.cls defaults to using nonbold math in the Abstract.
	% This preserves the distinction between vectors and scalars. However,
	% if the journal you are submitting to favors bold math in the abstract,
	% then you can use LaTeX's standard command \boldmath at the very start
	% of the abstract to achieve this. Many IEEE journals frown on math
	% in the abstract anyway. In particular, the Computer Society does
	% not want either math or citations to appear in the abstract.
	
	% Note that keywords are not normally used for peerreview papers.
	
	% make the title area
	\maketitle
	
	
	% To allow for easy dual compilation without having to reenter the
	% abstract/keywords data, the \IEEEcompsoctitleabstractindextext text will
	% not be used in maketitle, but will appear (i.e., to be "transported")
	% here as \IEEEdisplaynotcompsoctitleabstractindextext when compsoc mode
	% is not selected <OR> if conference mode is selected - because compsoc
	% conference papers position the abstract like regular (non-compsoc)
	% papers do!
	%\IEEEdisplaynotcompsoctitleabstractindextext
	% \IEEEdisplaynotcompsoctitleabstractindextext has no effect when using
	% compsoc under a non-conference mode.
	
	
	% For peer review papers, you can put extra information on the cover
	% page as needed:
	% \ifCLASSOPTIONpeerreview
	% \begin{center} \bfseries EDICS Category: 3-BBND \end{center}
	% \fi
	%
	% For peerreview papers, this IEEEtran command inserts a page break and
	% creates the second title. It will be ignored for other modes.
	%\IEEEpeerreviewmaketitle
	
	
	
	\section{}
	Note: here we use notation (x,y) to mean that the payoff for player I for the given strategy is x, and the payoff for player II is y. (LateX didn't want to cooperate)
	
	We are presented with the following payoff matrix
	\begin{table}[h]
		\centering
		\begin{tabular}{c|c|c|}
			
			& swerve & straight \\
			\hline
			swerve &(0, 0) & (-2, 2) \\
			\hline
			straight & (2, -2) & (-30, -10) \\
			\hline
		\end{tabular}
	\end{table}
	
	Then for Player 1:
	\[0 \cdot z_{11} + (-2)\cdot z_{12} \geq 2\cdot z_{11} + (-30) \cdot z_{12}  \]
	\[ 2 \cdot z_{21} + (-30) \cdot z_{22} \geq 0 \cdot z_{21} + (-2) \cdot z_{22}  \]
	
	And for Player 2:
	\[ 0 \cdot z_{11} + (-2) \cdot z_{21} \geq 2 \cdot z_{11} + (-10) \cdot z_{21}  \]
	\[ 2 \cdot z_{12} + (-10) \cdot z_{22} \geq 0 \cdot z_{12} + (-2) \cdot z_{22} \]
	
	With
	\[ z_{11} + z_{12} + z_{21} + z_{22} =1 \]
	\[ z_{11}, z_{12}, z_{21}, z_{22} \geq 0  \]
	
	We can express this as a linear programming problem:
	\begin{align*}
		\text{Maximise: } & 0\cdot z_{11} + 0 \cdot z_{12} + 0 \cdot z{21} + z_{22} &\\
		\text{subject to: } & &\\
		 -&2\cdot z_{11} + 28\cdot z_{12} \geq 0\\
		& 2 \cdot z_{21} - 28 \cdot z_{22}  \geq 0\\
		 -&2 \cdot z_{11} + 8 \cdot z_{21} \geq 0\\
		& 2 \cdot z_{12} -8 \cdot z_{22} \geq 0 \\
	\end{align*}
	
	This gives the following probabilities:
	
	\begin{table}[h]
		\centering
		\begin{tabular}{c|c|c|}
			
			& swerve & straight \\
			\hline
			swerve & 0 & $\frac{4}{19}$ \\
			\hline
			straight & $\frac{14}{19}$ & $\frac{1}{19}$ \\
			\hline
		\end{tabular}
	\end{table}
	And the probability of collision is $\dfrac{1}{19}$. 
	
	To check that this is indeed a correlated equilibrium, we substitute the values into the inequalities:
	
	\begin{align*}
		0 - 2\cdot \dfrac{4}{19} \geq 0 - 30 \cdot \dfrac{4}{19}  &\Longrightarrow -2 \cdot \dfrac{4}{19} \geq -30 \cdot \dfrac{4}{19}  \\
		2\cdot \dfrac{14}{19} -30 \cdot \dfrac{1}{19} \geq -2 \cdot \dfrac{1}{19} & \Longrightarrow 2 \cdot \dfrac{14}{19} \geq 28 \cdot \dfrac{1}{19}  \\
		0 - 2 \cdot \dfrac{14}{19}  \geq 2 \cdot 0 - 8 \cdot \dfrac{14}{19} &  \Longrightarrow -2 \cdot \dfrac{14}{19} \geq -8 \cdot \dfrac{14}{19} \\
		2 \cdot \dfrac{4}{19} -10 \cdot \dfrac{1}{19} \geq 0 - 2 \cdot \dfrac{1}{19} & \Longrightarrow 2 \cdot \dfrac{4}{19} \geq 8 \cdot \dfrac{1}{19}  \\
	\end{align*}
	All the inequalities hold, so we have found a correlated equilibrium. 
	%\IEEEPARstart{}{} 
	

	
	% that's all folks
\end{document}

