\documentclass[11pt,journal]{article}
%\usepackage{hyperref}
%\usepackage[breaklinks]{hyperref}
\usepackage{breakurl}
\usepackage{url}
\usepackage{listings}
\usepackage{courier}
\usepackage{amsmath}
%\ifCLASSOPTIONcompsoc
% IEEE Computer Society needs nocompress option
% requires cite.sty v4.0 or later (November 2003)
\usepackage[nocompress]{cite}

%\else
% normal IEEE
\usepackage{cite}
%\fi

\hyphenation{op-tical net-works semi-conduc-tor}
\addtolength{\oddsidemargin}{-.875in}
\addtolength{\evensidemargin}{-.875in} 
\addtolength{\textwidth}{1.75in}

\addtolength{\topmargin}{-.875in}
\addtolength{\textheight}{1.75in}

\begin{document}
	\title{Algorithmic Game Theory, Assignment 1}
	
	\author{Agata~Borkowska,~UID: 1690550}% <-this % stops a space
		%\protect\\
		%\thanks{}}
	
	% The paper headers



	% IEEEtran.cls defaults to using nonbold math in the Abstract.
	% This preserves the distinction between vectors and scalars. However,
	% if the journal you are submitting to favors bold math in the abstract,
	% then you can use LaTeX's standard command \boldmath at the very start
	% of the abstract to achieve this. Many IEEE journals frown on math
	% in the abstract anyway. In particular, the Computer Society does
	% not want either math or citations to appear in the abstract.
	
	% Note that keywords are not normally used for peerreview papers.
	
	% make the title area
	\maketitle
	
	
	% To allow for easy dual compilation without having to reenter the
	% abstract/keywords data, the \IEEEcompsoctitleabstractindextext text will
	% not be used in maketitle, but will appear (i.e., to be "transported")
	% here as \IEEEdisplaynotcompsoctitleabstractindextext when compsoc mode
	% is not selected <OR> if conference mode is selected - because compsoc
	% conference papers position the abstract like regular (non-compsoc)
	% papers do!
	%\IEEEdisplaynotcompsoctitleabstractindextext
	% \IEEEdisplaynotcompsoctitleabstractindextext has no effect when using
	% compsoc under a non-conference mode.
	
	
	% For peer review papers, you can put extra information on the cover
	% page as needed:
	% \ifCLASSOPTIONpeerreview
	% \begin{center} \bfseries EDICS Category: 3-BBND \end{center}
	% \fi
	%
	% For peerreview papers, this IEEEtran command inserts a page break and
	% creates the second title. It will be ignored for other modes.
	%\IEEEpeerreviewmaketitle
	
	
	
	\section{}
	Note: here we use notation (x,y) to mean that the payoff for player I for the given strategy is x, and the payoff for player II is y. (LateX didn't want to cooperate)
	\subsection{}
	Let $b = 1$, $a = 1$, $c=-1$, $d=1$. i.e.
	\begin{table}[h]
		\centering
		\begin{tabular}{c|c|c}
			& A & B \\
			\hline
			1 &(1,0) & (0,1) \\
			\hline
			2 & (0, 0) & (1, -1) \\
		\end{tabular}
	\end{table}
	
	Then:
	
	\begin{itemize}
		\item If player I chooses 1, then player II would want to play B, but...
		\item if player II chooses B, player I will want to play 2, but...
		\item if player I chooses 2, player II will want to play A, but...
		\item if player II chooses A, player I will want to choose 1.
	\end{itemize}

	There is a mixed Nash equilibrium, in which each player plays each strategy with a probability $\dfrac{1}{2}$. The expected payoff for I is $\dfrac{1}{2}$, and for II is 0.
	
	\subsection{}
	Assuming Player I has a strategy $(\dfrac{1}{2}, \dfrac{1}{2})$, the payoff for Player II if they chose B would be $\dfrac{1}{2} \cdot a + \dfrac{1}{2} \cdot c$, and the payoff if they chose A is 0.
	
	Therefore, for Player II to always prefer A over B, \[\dfrac{1}{2} \cdot a + \dfrac{1}{2} \cdot c < 0\]
	\[\Rightarrow a+c <0\]
	
	\subsection{}
	Let $a,b,c,d = 0$. Then for either player, a strategy $(\lambda, 1-\lambda)$ for any $\lambda \in [0,1]$, the payoff is 0, and there are infinitely many such $\lambda$'s.
	
	\subsection{}
	Let $b = 1$, $a = -1$, $c=1$, $d=1$. I.e.
	
		\begin{table}[h]
		\centering
		\begin{tabular}{c|c|c}
			& A & B \\
			\hline
			1 &(1,0) & (0,-1) \\
			\hline
			2 & (0, 0) & (1, 1) \\
		\end{tabular}
	\end{table}
	Then the game has 2 pure Nash equilibria: (1, A) and (2, B). If I chooses 1, then II choosing A always gives them a better payoff, because $a<0$. If II chooses A, I will choose 1, because $b>0$. Similarly for (2, B).

	
	\section{}
	The possible choices for player I are 2, 3, and 4, and for player II are \{2,3\}, \{2,4\}, and \{3,4\}. Putting it in a payoff matrix (note that this is a 0-sum game, and we only need to write the payoff for player I, as player II will get 0 - that amount).
	
			\begin{table}[h]
		\centering
		\begin{tabular}{c|c|c|c}
			& \{2,3\} & \{2,4\} & \{3,4\} \\
			\hline
			2 & -2 & -2 & 0 \\
			\hline
			3 & -3 & 0 & -3 \\
			\hline
			4 & 0 & -4 &  -4 \\
		\end{tabular}
	\end{table}

	Let player I play 2 with probability $p$, 3 with probability $q$, and 4 with probability $(1-p-q)$. Then II's expected winnings are:
	
	\begin{itemize}
		\item \textbf{II plays \{2,3\}}:  $2 \cdot p +3 \cdot q + 0 \cdot (1-p-q) = 2\cdot p + 3 \cdot q$
		\item \textbf{II plays \{2,4\}}:  $2 \cdot p +0 \cdot q + 4 \cdot (1-p-q) = (-2)\cdot p + (-4) \cdot q + 4$
		\item \textbf{II plays \{3,4\}}:  $0 \cdot p +3 \cdot q + 4 \cdot (1-p-q) = (-4) \cdot p -q + 4$
	\end{itemize}

	For player II to be willing to randomise, the expected payoffs for each of those strategies should be the same, call it $V$.
	
	Thus we arrive at a system of linear equations with 3 unknowns:
	\begin{align}
	2\cdot p + 3\cdot q - V&= 0 \\
	(-2)\cdot p+ (-4) \cdot q - V &= -4 \\
	(-4)\cdot p - q -V&= -4 
	\end{align}
	After solving it, we get:
	\[p=\dfrac{6}{13}\approx 0.462 \]
	\[q =\dfrac{4}{13}\approx 0.307 \]
	\[(1-p-q) = \dfrac{3}{13} \approx0.231 \]
	So player I's strategy is $(\dfrac{6}{13}, \dfrac{4}{13}, \dfrac{3}{13})$.
	
	Now let's assume player II picks \{2,3\} with probability $s$, \{2,4\} with probability $t$, and \{3,4\} with probability $1-s-t$. Let the winnings for player I be $W$. The system of linear equations is now:
	
	\begin{align}
	-2\cdot s -2 \cdot t -W & =0 \\
	3 \cdot t - W& = 3\\
	4\cdot s- W& = 4
	\end{align}
	after solving it, we get:
	\[s=\dfrac{7}{13}\approx 0.538 \]
	\[t =\dfrac{5}{13}\approx 0.385 \]
	\[(1-s-t) = \dfrac{1}{13} \approx0.077 \]
	
	So player II's strategy is $(\dfrac{7}{13}, \dfrac{5}{13}, \dfrac{1}{13})$.
	
	As a useful check, from the above systems of linear equations, we get $V = -W = \dfrac{24}{13} \approx 1.846$. This is a zero sum game, and the expected payoff for both players adds up to 0.
	
	Another way of finding Nash equilibrium would be to use linear programming, however in a simple game like this one, this method suffices.
	
	%\IEEEPARstart{}{} 
	

	
	% that's all folks
\end{document}

